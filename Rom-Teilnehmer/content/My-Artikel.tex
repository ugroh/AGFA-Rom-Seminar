% !TEX root = ../My-Rom-Beitrag.tex

%% -- Stand: 2022/10/15
%% --


%% --Einige Definitionen, damit die Eingabe einfacher wird
%% --
\newcommand{\LongTitel}{Hier steht der lange Titel des Beitrags}
\newcommand{\ShortTitel}{Hier steht der kurze Titel}
\newcommand{\AutorenBeitrag}{Autor1, Autor2 \ldots}


%% -- Titelseite des Beitrags
%% --
\addchap{\LongTitel}

%% -- Für Kopfzeile und TOC Eintrag
%% --
\markleft{\textsc{\AutorenBeitrag}}		% auf den geraden Seiten erscheinen die Autoren
\markright{\textsc{\ShortTitel}}		% auf den ungeraden Seiten der kurze Titel
%% 							
\addtocontents{toc}{\textsc{\AutorenBeitrag}}
\addtocontents{toc}{}

%% -- Titelseite des Beitrags
%% --
\begin{center}
\textsc{\Large \AutorenBeitrag}
\end{center}
	\vspace{1cm}
\begin{center}
	\includegraphics[width=8cm, height=8cm, keepaspectratio=true]{./content/My-Artikel-Cover}
\end{center}

%% -- Würde keine Extraseite daraus machen
%% --
%\newpage

%% --
%% -- ab hier die Sammlung der LIT-Zitate
\begin{refsection}

%% -- Begin des eigenen Artikels
%% --
\section*{Eine erste Überschrift}
% 
Da es sich um kurze Beiträge handelt, kann man aus meiner Sicht auf Unterverzeichnisse via
\verb|\subsection{Überschrift}| verzichten.
Wer es dennoch machen will bitte beachten, dass ein solcher Abschnitt dann mindestens drei Absätze enthalten sollte, die wiederum aus mindestens fünf Sätzen bestehen.

Bei der Eingabe bitte folgendes beachten.

\begin{itemize}[nosep, topsep=1em]
\item
Text, der in Hochkommata (Gänzefüßchen) eingeschlossen ist bitte mittels \verb|\enquote{Text}|
eingegeben, \dh \enquote{Text}; dann erscheinen auch die richtigen.

\item
Auszeichnungen werden nicht fett mittels \verb|\textbf{Text}|, sondern mit \verb|\emph{Text}|
eingegeben, \dh \emph{Text}.
Bitte nie \verb|\textit{Text}| verwenden.

\item
Es ist ein Unterschied, ob man d.h. schreibt oder korrekt \dh
Für die Abkürzungen mal in die Datei \texttt{Rom-Abkuerzungen.tex} reinsehen und auch den Duden konsultieren.

\item
Für Aufzählungen, \dh Listen, hat sich das Paket \texttt{enumitem} mit den vielen Möglichkeiten bewährt.
Will man etwa römische kleine Zahlen haben: \verb|\begin{enumerate}[(i)]| macht es.

\begin{enumerate}[(i), nosep]
\item
Etwa hier und dann
\item
Hier nochmals
\end{enumerate}

Einfach mal diese Datei im Original ansehen.
Vor allem, wie man die Listen kompakter bekommt.

\item
Literatur kann man einfach mit der \LaTeX{} Umgebung \texttt{thebibliography} eingeben oder mit Hilfe von  \texttt{bibtex} und \texttt{biblatex}.
Auf jeden Fall sollten die Zitate korrekt sein, \dh bei Verweise auf die mathematsiche Artiekl etwa die Abkürzungen der Zeitschriften.
Dazu bitte das \emph{Zentralblatt für Mathematik} nutzen; siehe \href{https://zbmath.org}{https://zbmath.org}.
Weiters findet sich in meinem \LaTeX-Tipp zur Literaturverwaltung. 

\end{itemize}

Der nun folgende Text stammt nicht von mir, da ich kein Schachspieler bin sondern Go bevorzuge.
Zu Go siehe etwa \href{https://bit.ly/3CEdJom}{diesen Artikel auf Wikipedia}.

Literatur zu \TeX{}: \textcite{kohm:2020}, \textcite{voss:2012a,voss:2017b}, \textcite{latextipps5}.

\subsection{Etwas zu Schach}

Schach ist ein Brettspiel und wird auf einem quadratischen Brett gespielt. Dies ist in 64 Felder eingeteilt.

Am Anfang bekommt jeder Spieler 16 Figuren. Dies sind:
\begin{enumerate}[label=\textbullet, nosep]  % enger zusammen
  \item 
  acht Bauern,
  \item 
  zwei Türme,
  \item 
  zwei Springer,
  \item 
  zwei Läufer,
  \item 
  eine Dame,
  \item 
  und ein König.
\end{enumerate}

\subsection{Eröffnung}
Den Beginn einer Schachpartie nennt man Eröffnung. 
Diese sind zumeist sehr ausführlich analysiert. 
Viele Schachprogramme haben eine Eröffnungsbibliothek mit mehreren tausend Zügen. 
Eine Eröffnung könnte beispielsweise so aussehen
%
\begin{enumerate}[--, nosep]
  \item e4 e5
  \item Lc4 d6
  \item Df3 Sc6
  \item Dxf6\#
\end{enumerate}

und man nennt diese Eröffnung das \emph{Schäfermatt} (Zugfolge hier ist aber falsch)

%%% --------------------------------
%%%  -- Was
%%% --------------------------------
%\par\medskip
%\begin{minipage}[c][1.0\height][r]{.95\linewidth}
%\begin{center}
%\newchessgame
%\mainline{1. e4 e5 2. Bc4 d6 3. Qh5 Nf6 4. Qf7}
%\xskakset{moveid=3w}%
%\chessboard[setfen=\xskakget{nextfen}]\\[1ex]
%Position nach Zug 3 \, \xskakget{lan}
%\end{center}
%Dann zieht Weiß auf f7 und Schwarz ist matt.
%Wer es genauer wissen will, bitte \href{https://ctan.org/pkg/xskak}{https://ctan.org/pkg/xskak} nachsehen \bzw Manual lesen. 
%\end{minipage}
%\par\medskip
%% ----------------------------
%


\subsection{Verlauf des Spieles}
%
Ein Schachspiel unterteilt sich in drei Abschnitte:
%
\begin{description}
  \item[Eröffnung] 
  Dies ist der Anfang einer Partie. 
  Man versucht, zunächst alle Figuren optimal zu platzieren.
  
  \item[Mittelspiel] 
  Hier kommt es sehr auf Strategie und Taktik an. 
  Jeder Spieler versucht, das Spiel zu seinen Gunsten zu steuern.
  
  \item[Endspiel] 
  Wenn dann nur wenige Figuren auf dem Brett verblieben sind, spricht man vom Endspiel. 
  Meist wird versucht, einen Bauern in eine höherwertige Figur umzuwandeln und den König Matt zu setzen.
\end{description}
%%
Die Rangliste sieht wie folgt aus (etwas veraltet):
\begin{center}
  \begin{tabular}{llc}
    Rang & Name              & Rating \\\hline
    1    & Garry Kasparov    & 2817   \\
    2    & Viswanathan Anand & 2774   \\
    3    & Wladimir Kramnik  & 2764
  \end{tabular}
\end{center}
%% \vref bitte verwenden
Literatur hierzu kann nachgelesen in \textcite{brunel:1999}. 
Zum Schluß noch ein kleines Bild, siehe \vref{fig:label1} und etwas \textbf{fette} oder \emph{kursive} Schrift.

\subsection{Unterabschnitt}
%%
\blindtext[2]

%%
\section*{Ein neuer Hauptabschnitt}
\subsection{Mit einem neuen Unterabschnitt}
\lipsum[1-3]

\subsection{Noch ein Unterabschnitt}
\lipsum[4-6]

\begin{figure}[t]
  \begin{center}
  \includegraphics[width=55mm]{./content/My-Artikel-Cover}
  \caption{Bildunterschift 1}\label{fig:label1}		%\vspace{-3mm}
  \end{center}
\end{figure}

%% --
%% -- LitVerzeichnis des Beitrags
%\nocite{*}
% \addbibresource{./bib/name.bib}
\printbibliography
\end{refsection}

%\bibliography{article} -> falsch siehe Masterfile
