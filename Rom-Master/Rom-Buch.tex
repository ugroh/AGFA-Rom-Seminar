% !TEX TS-program = pdflatexmk
%% -- Stand: 2022/12/30
%% -- Wir nutzen KOMA-Script
%% --
\documentclass[%
	,ngerman					% Deutsch als Standard, andere über babel
	,bibliography	= notoc		% siehe KOMA-Script 3.23, Seite 174
	,bibliography 	= leveldown % kein eigenes Kapitel sondern ein Abschnitt
	,parskip = half+			% halbe Leerzeile Abstand für Absätze
	]{scrbook}

%% -- KOMA Otions
%% -- 
\KOMAoptions{%
	,paper 		= a4
	,BCOR		= 15mm		% siehe KOMA-Script 2.6, S. 39
	,twoside	= true	% siehe KOMA-Script 2.6, S. 47
	,fontsize	= 12pt		% Standard ist 11pt
%	,DIV		= 12		% siehe KOMA-Script 2.6, S. 42
	,open		= any		% siehe KOMA-Script 3.16, S. 111
	,headings 	= normal	% siehe KOMA-Script 3.16, S. 113
	}

%% -- Gehe davon aus, dass alle UTF8-nutzen
%% --
\usepackage[T1]{fontenc}		%% für LuaLaTeX entbehrlich


%% -- Sprache und Anführungsszeichen richtig
%% -- 
\usepackage[english, italian, main=ngerman]{babel} 
%% -- Andere Sprache via \selectlanguage{sprache}; siehe Manual

%% -- Hochkomma: Eingabe mittels \enquote{text}; dann wird alles richtig
\usepackage[%
	,autostyle
	,italian= guillemets
	, german=guillemets
	, english=american]{csquotes}

%% -- Die Schrift
%% --
\usepackage{libertine}

%% -- Das Layout
%% --
%% --
%% -- Rombuch Layout
%% -- Stand 2022/12/30
%% --

%% -- Alles etwas kleiner
%% --
\setkomafont{chapter}{\LARGE}
\setkomafont{section}{\large}
\setkomafont{subsection}{\normalsize}

%% -- Keine section etc. im TOC
%% --
\addtocounter{tocdepth}{-2}% Keine \section im TOC

%% -- chapter
%% --
%\addtokomafont{chapterentrypagenumber}{\color{white}}
\renewcommand*{\chapterformat}
 {\chapappifchapterprefix{\ }\thechapter\autodot\enskip}

\renewcommand*\raggedchapter{\centering}	% KOMA: Kapitelüberschriften zentrieren
	%
\RedeclareSectionCommand[%
  ,beforeskip	= -1.5\baselineskip
  ,afterskip	= 2.5\baselineskip
  ,innerskip	= 0.75ex
  ,tocnumwidth	= 7em
]{chapter}
%
\renewcommand*{\addchaptertocentry}[2]{%  Seite 524 / http://www.komascript.de/node/1780
  \ifstr{#1}{}{% keine Nummer:
    \addtocentrydefault{chapter}{#1}{#2}% wie bisher
  }{% mit Nummer:
    \addtocentrydefault{chapter}{}{#2} %\chapapp\ #1.-
  }%
}

%% -- section
%% --
\renewcommand*{\thesection}{\arabic{section}} 

\RedeclareSectionCommand[tocnumwidth=2em]{section} % TOC
\renewcommand*{\addsectiontocentry}[2]{% Eintrag in TOC
  \addtocontents{toc}{\protect\addvspace{\protect.5\baselineskip}}% Halbe LZ
  \Ifstr{#1}{}{%
    \addtocentrydefault{section}{#1}{#2}%
  }{%
    \addtocentrydefault{section}{#1.}{\itshape #2}
  }%
}

%% -- subsection
%% --
\renewcommand*{\thesubsection}{\arabic{subsection}}
\renewcommand*{\subsectionformat}{\thesubsection.\enskip}% KOMA: Nummer in der Überschrift
\RedeclareSectionCommand[%
  ,beforeskip	= -1.75\baselineskip
  ,afterskip	= 0.25\baselineskip
  ,font			= \itshape
  ,tocnumwidth	= 1em
]{subsection}
%% -- Eintrag in das TOC
\renewcommand*{\addsubsectiontocentry}[2]{%
  \Ifstr{#1}{}{%
    \addtocentrydefault{subsection}{#1}{#2}%
  }{%
    \addtocentrydefault{subsection}{#1.}{#2}% Mit Punkt hinter der Nummer in TOC
  }%
}

%% -- \subsubsection: Nur Nummer 1. etc; i.a. keine Überschrift sondern es 
%% -- geht mit dem Text weiter. 
%% -- Wer es ohne die Nummer will, aber mit einer Überschrift, dann die 
%% -- Sternvariante \subsubsection*{} nutzen. 
%% -- Siehe hierzu das Musterfile für Beispiele
%% --
\setcounter{secnumdepth}{\subsubsectionnumdepth} 	% subsub-nummer
\renewcommand*{\thesubsubsection}{\arabic{subsubsection}.\kern3pt} % 1.
\RedeclareSectionCommand[%
  	,beforeskip	=  	.25\baselineskip	% Skip vor dem Kommando + indent
	,afterskip	= 	-0.5em				% Kein Skip aber Abstand
	,font		=	\itshape			% Falls Überschrift 
%	,tocnumwidth=	1.25em				% Eintrag in TOC ist aber unterbunden
%	,tocindent	= 1em					%  
	]{subsubsection} 
	
%% -- Kopf- und Fußzeile
%% --
\RequirePackage[automark]{scrlayer-scrpage}

\clearpairofpagestyles
\renewcommand*{\chapterpagestyle}{empty}						% Nichts auf der Seite mit Kapitel

%% -- Kopfzeile
%% --
\KOMAoptions{headsepline=0.8pt}
\renewcommand*{\sectionmarkformat}{}		% keine Abschnittsnummer
\renewcommand*{\subsectionmarkformat}{}	% keine UnterAbschnittsnummer

%% ---------------------------
%% Zweiseitig
%% --------------------------
\manualmark
\ihead{\headmark} 	%% -- innen

\rohead{%% außen gerade
  \makebox[0pt][l]{%
    \makebox[\marginparsep][r]{%
      \rule[-\dp\strutbox]{1pt}{\normalbaselineskip}\nobreakspace
    }%
    \makebox[\marginparwidth][l]{\pagemark}%
  }}

\lehead{%% außen ungerade
  \makebox[0pt][r]{%
    \makebox[\marginparwidth][r]{\pagemark}%
    \makebox[\marginparsep][l]{%
      \nobreakspace\rule[-\dp\strutbox]{1pt}{\normalbaselineskip}%
    }}}

\pagestyle{scrheadings}


%% --Fußnoten Nach KOMA 3.14 
%% --
\RequirePackage[newcommands,footnotes,raggedrightboxes]{ragged2e}
\deffootnotemark{\kern1pt(\raisebox{0.35ex}{\footnotesize\thefootnotemark})}
\deffootnote[1.5em]{1em}{2em}{(\raisebox{0.35ex}{\footnotesize\thefootnotemark})\kern5pt}
\let\raggedfootnote\RaggedRight
%% %%%%%%%%%%%%%%%%%%%%%%%%%%%%%%%%%%%


%\renewcommand*{\figureformat}{}	% Die Bezeichnung Abbildung kann entfallen
									% Nur wer auf Abbildung verweist, % weg

%% -- Die Pakete
%% --
%% -- Die Pakete für RomSeminar
%% -- Stand: 2022/10/13
%% --

%% -- Mathematik
\usepackage{%
	,amsmath
	,amssymb
	,amsthm} 
\usepackage[tbtags]{mathtools}

%% -- tikz
\usepackage{tikz}
\usetikzlibrary{positioning,arrows}

%% --
\usepackage[newcommands,footnotes,raggedrightboxes]{ragged2e}
	
\usepackage{% Voss, LaTeX - 6.3 - Seite 227ff					
	,array 			   	
	,booktabs
	,tabularx
	,wrapfig
	,longtable}
	
\RequirePackage{% Voss, LaTeX - 8.6 - Seite 340ff					
	,caption 			   	
	,floatrow
	,subcaption}

\RequirePackage{% 						
	,multicol	% Voss, LaTeX - 5.15.1 - Seite 215	
	,parallel	% Voss, LaTeX - 5.15.2 - Seite 216
	}
	 	   	
\RequirePackage{%
	,graphicx	% Voss, LaTeX - 5.10.2 - Seite 174
	,wrapfig		% Voss, LaTeX - 5.10.3 - Seite 177
	,cutwin		% Voss, LaTeX - 5.10.3 - Seite 178
	}

%% -- Aufzählungen
\usepackage[%
	,inline			% für Aufzählungen im Text = Sternvariante
	,shortlabels		% damit \begin{enumerate}[(i)] funktioniert
	]{enumitem}

%% -- Wenn schon, dann muss man es auch definieren
\usepackage{nicefrac}

%% -- Zum Testen
\usepackage{blindtext,lipsum}

%% -- Für das Schachspiel
\usepackage{xskak}


%% -- Falls erforderlich, sind halt alte. Pakete
%% -- 
\usepackage{dialogue}	% gibt es sicherlich Nachfolger
\usepackage{attrib}		% dito

\usepackage[nospace]{varioref}				% \vref
% -----------------------------------------------------------
\RequirePackage[%            	%  texdoc hyperref; zum Nutzen der Hyperlinks;                                                                                                        
	,colorlinks=true    			%  Farbige Links false/true, für onlineversion true                                                           
	,urlcolor=blue       		%                                                              
	,citecolor=blue      		%                                                              
	,linkcolor=blue      		%    
	,breaklinks= true                                                     
	]{hyperref}          		%% \href{xterner-Link}{Bezeichnung-in-Doku}
% -----------------------------------------------------------
\usepackage{cleveref}


% \hypersetup{hidelinks}					% Vor dem Druck % entfernen

%% -- Die Abkürzungen
%% --
%% -- Richtige Interpunktion
%% --
\usepackage{xspace}
%%
%% -- Richtige Abkürzungen
% --> Englisch
\newcommand{\eg}{e.g.\xspace}
\newcommand{\ie}{i.e.\xspace}
% weitere hier eintragen

% --> Deutsch
\newcommand{\zB}{\mbox{z.\,B.}\xspace}
\newcommand{\iA}{\mbox{i.\,A.}\xspace}
\newcommand{\iAllg}{\mbox{i.\,Allg.}\xspace}
\newcommand{\di}{\mbox{d.\,i.}\xspace}
\renewcommand{\dh}{\mbox{d.\,h.}\xspace}
\newcommand{\oAe}{\mbox{o.\,Ä.}\xspace}
\newcommand{\uAe}{\mbox{u.\,Ä.}\xspace}
\newcommand{\og}{\mbox{o.\,g.}\xspace}
\newcommand{\ua}{\mbox{u.\,a.}\xspace} 
\newcommand{\lc}{\mbox{l.\,c.}\xspace} 
%%
\newcommand{\inkl}{inkl.\xspace} 
\newcommand{\sog}{sog.\xspace} 
\newcommand{\bzgl}{bzgl.\xspace} 
\newcommand{\fue}{\mbox{f.\,ü.}\xspace}
\newcommand{\vs}{vs.\xspace} 
\newcommand{\ca}{ca.\xspace}
\newcommand{\bzw}{bzw.\xspace}
\renewcommand{\etc}{etc.\xspace}
\newcommand{\usw}{usw.\xspace}
\newcommand{\ggf}{ggf.\xspace}
\newcommand{\evtl}{evtl.\xspace}

%% -- Literatur
%% -- Die einzelnen Referenzen kann man lokal halten
%% -- \begin{refsection} .. \printbibliography \end{refsection} 
%% --
%% --
%% Makros für biblatex
%% Stand: 2023/01/03
%% --
\usepackage[%
  	,style 		= numeric	
   	,sorting		= nyt
	,hyperref	= true  
 	,maxnames	= 4
   	,minnames	= 3
   	,giveninits	= true		% Vornamen als Initiale Ulrich -> U.
	,backend		= biber
 	,uniquename	= false
]{biblatex}  	
%% --
%\setlength{\bibhang}{1em} 
%\setlength{\bibnamesep}{.5\normalbaselineskip}
%\setlength{\biblabelsep}{.75em}
%% --
\AtEveryBibitem{\clearfield{issn}}%
\AtEveryBibitem{\clearlist{language}}%
\AtEveryBibitem{\clearlist{location}}%
\AtEveryBibitem{\clearfield{pagetotal}}%
\AtEveryBibitem{\clearfield{month}}%
%% %%%%%%%%%%%%%%%%%%%%%%%%%%%%%%%%%%%%%%%%%%%%%%%%%%%%%
%% Kein Komma nach Verlag bei Büchern
%% %%%%%%%%%%%%%%%%%%%%%%%%%%%%%%%%%%%%%%%%%%%%%%%%%%%%%
\renewbibmacro*{publisher+location+date}{%
  \printlist{publisher}%
  \setunit*{\addspace}%
%  \printlist{location}%
%  \setunit*{\addsemicolon\space}%
  \usebibmacro{date}%
  \newunit%
	} % ende makro
%% %%%%%%%%%%%%%%%%%%%%%%%%%%%%%%%%%%%%%%%%%%%%%%%%%%%%%
%% Ausgabe Journal-Band(Nummer)-Jahr-Seiten
%% %%%%%%%%%%%%%%%%%%%%%%%%%%%%%%%%%%%%%%%%%%%%%%%%%%%%%	
\renewbibmacro*{journal+issuetitle}{% 
  \usebibmacro{journal}% 
  \setunit*{\addspace}% 
  \iffieldundef{series} 
    {} 
    {\newunit 
     \printfield{series}% 
     \setunit{\addspace}}% 
  \printfield{volume}% 
  \iffieldundef{number} 
     {} 
      {\kern1pt\mkbibparens{\printfield{number}}}% \addspace
  \setunit{\addcomma\space}% 
  \printfield{eid}% 
  \setunit{\addspace}% 
  \usebibmacro{issue+date}% 
  \setunit{\addcolon\space}% 
  \usebibmacro{issue}% 
  \newunit}
%% %%%%%%%%%%%%%%%%%%%%%%%%%%%%%%%%%%%%%%%%%%%%%%%%%%%%%
%% Felder bei Typ -- article --
%% %%%%%%%%%%%%%%%%%%%%%%%%%%%%%%%%%%%%%%%%%%%%%%%%%%%%%
\DeclareFieldFormat[article]{number}{#1} 
\DeclareFieldFormat[article]{volume}{\textbf{#1}} 
\DeclareFieldFormat[article]{pages}{#1}
\DeclareFieldFormat[article]{title}{\mkbibemph{#1}}
\DeclareFieldFormat{journaltitle}{#1}
%% %%%%%%%%%%%%%%%%%%%%%%%%%%%%%%%%%%%%%%%%%%%%%%%%%%%%%
%% Felder bei Typ -- book --
%% %%%%%%%%%%%%%%%%%%%%%%%%%%%%%%%%%%%%%%%%%%%%%%%%%%%%%
\DeclareFieldFormat[book]{title}{\mkbibemph{#1}}
\DeclareFieldFormat[book]{date}{\mkbibparens{#1}}
\DeclareFieldFormat[book]{pages}{} 
\DeclareFieldFormat[book]{pagetotal}{} 
\DeclareFieldFormat[book]{url}{} 
\DeclareFieldFormat[book]{language}{}
\DeclareFieldFormat[book]{isbn}{} 
\DeclareFieldFormat[book]{series}{} 
\DeclareFieldFormat[book]{number}{} 
\DeclareFieldFormat[book]{edition}{} 
\DeclareFieldFormat[book]{address}{} 
%% %%%%%%%%%%%%%%%%%%%%%%%%%%%%%%%%%%%%%%%%%%%%%%%%%%%%%
%% Felder bei Typ -- manual --
%% %%%%%%%%%%%%%%%%%%%%%%%%%%%%%%%%%%%%%%%%%%%%%%%%%%%%%
\DeclareFieldFormat[manual]{title}{\mkbibemph{#1}\addspace}
\DeclareFieldFormat[manual]{date}{}
\DeclareFieldFormat[manual]{version}{}
%% %%%%%%%%%%%%%%%%%%%%%%%%%%%%%%%%%%%%%%%%%%%%%%%%%%%%%
\DeclareFieldFormat{postnote}{#1}
\DeclareFieldFormat{multipostnote}{#1}
%%%
\renewcommand*{\labelnamepunct}{\addcolon\space} 		% : nach Namen oder \addcomma
\renewcommand*{\finalnamedelim}{\addspace\&\addspace} 	% kein "und" bei Doppelname
\renewcommand*{\bibpagespunct}{\addspace} % : vor Seiten oder \addcomma
%%\renewcommand*{\newunitpunct}{\addspace}
%%\renewcommand*{\finentrypunct}{}
\renewcommand{\mkbibnamegiven}{\textsc}
\renewcommand{\mkbibnamefamily}{\textsc}
%%%
\renewcommand*{\intitlepunct}{}
\renewbibmacro{in:}{}
%%
%% %%%%%%%%%%%%%%%%%%%%%%%%%%%%%%%%%%%%%%%%%%%%%%%%%%%%%
%% URL 
%% %%%%%%%%%%%%%%%%%%%%%%%%%%%%%%%%%%%%%%%%%%%%%%%%%%%%%
%%
\DefineBibliographyStrings{german}{%
	andothers = {et al. },
	and     = {u\adddot},
  	urlseen = {aufgerufen am },
  	urlfrom = {online unter },
	}
\DeclareFieldFormat{url}{\bibstring{urlfrom}\addcolon\space\url{#1}}
%% %%%%%%%%%%%%%%%%%%%%%%%%%%%%%%%%%%%%%%%%%%%%%%%%%%%%%
%% URL auf neuer Zeile
%% %%%%%%%%%%%%%%%%%%%%%%%%%%%%%%%%%%%%%%%%%%%%%%%%%%%%%
\newbibmacro*{bbx:parunit}{%
  \ifbibliography
    {\setunit{\bibpagerefpunct}\newblock
     \usebibmacro{pageref}%
     \clearlist{pageref}%
     \setunit{\par\nobreak}}
    {}}
%
\renewbibmacro*{doi+eprint+url}{%
  \usebibmacro{bbx:parunit}% Added
  \iftoggle{bbx:doi}
    {\printfield{doi}}
    {}%
  \iftoggle{bbx:eprint}
    {\usebibmacro{eprint}}
    {}%
  \iftoggle{bbx:url}
    {\usebibmacro{url+urldate}}
    {}}
%  
\renewbibmacro*{eprint}{%
  \usebibmacro{bbx:parunit}% Added
  \iffieldundef{eprinttype}
    {\printfield{eprint}}
    {\printfield[eprint:\strfield{eprinttype}]{eprint}}}
%%
%% %%%%%%%%%%%%%%%%%%%%%%%%%%%%%%%%%%%%%%%%%%%%%%%%%%%%
%% Url und DOI an Titel binden Voss 4.10.10, S. 200
%% und https://bit.ly/2YghEHH
%% %%%%%%%%%%%%%%%%%%%%%%%%%%%%%%%%%%%%%%%%%%%%%%%%%%%%%%
%%
\newbibmacro{string+doi}[1]{%
  \iffieldundef{doi}{#1}{\href{http://dx.doi.org/\thefield{doi}}{#1}}}
\DeclareFieldFormat{title}{% Titel+Formatierung an das Makro weiterreichen
  \usebibmacro{string+doi}{\mkbibemph{#1}}}
\DeclareFieldFormat[article,manual]{title}{\usebibmacro{string+doi}%
	{\mkbibemph{#1}}}
	



\addbibresource{./bib/Rom-Biblio.bib}	%% Eine Gesamtbibliothek ist besser
\ExecuteBibliographyOptions{%
	,backref		= false	% true = wo habe ich was referenziert
 	,url		= true		% online-Zitate werden separat gezeigt
 	,doi		= true		% DOI ist hinterlegt 
	,eprint		= false		% true Zitate werden separat gezeigt
	}

%% -- Wir starten nun
%% --
\begin{document}

%% -- Frontmatter
\frontmatter
% !TEX root = ../Rom-Buch.tex
% ---------------------
% Titelseite Rom-Titelseite.tex
% Stand	: 2022/10/13
% ---------------------
\begin{titlepage}
	\newcommand{\HRule}{\rule{\linewidth}{.25mm}}
	\vspace*{\stretch{1}}
	\HRule
	\vspace*{10pt}
	%\begin{flushright}
	\begin{center}
	  {\Huge\scshape Hard Problems \\[5mm]
				Romseminar 2022 \\ }
%	  	\enquote} \\[5mm]
%	  {\LARGE } \\[7.5mm]
	\vspace{2cm}
	\includegraphics[width=12cm, height=12cm, keepaspectratio=true]{./content/Rom-hard-problems-2022}
	
	  {Version: \today }
	\end{center}
	%\end{flushright}
	\HRule
	\vspace*{\stretch{2}}
	\begin{center}
%	  {\Large\scshape\textcopyright\ Ulrich Groh \\ Mössingen 2021}
	\end{center}
\end{titlepage}

\tableofcontents
% !TEX root = ../Rom-Buch.tex
%% --
%% -- Rom-Vorwort.tex
%% --
\chapter*{Ein kleines Vorwort}
\lipsum[1-2]
% !TEX root = ../Rom-Buch.tex
%% --
%% -- Rom-Agenda.tex
%% --
\chapter*{Die Agenda}
%% mit longtable setzen
Hier findet sich die Termine, Titel der Vorträge und was sonst noch so alles passiert ist

%% -- Mainmatter
\mainmatter

% !TEX root = ../Rom-Buch.tex
%% -- Stand: 2022/10/14
%% --


%% --Einige Definitionen, damit die Eingabe einfacher wird
%% --
\newcommand{\LongTitel}{Hier steht der lange Titel des Beitrags}
\newcommand{\ShortTitel}{Hier steht der kurze Titel}
\newcommand{\AutorenBeitrag}{Autor1, Autor2 \ldots}


%% -- Titelseite des Beitrags
%% --
\addchap{\LongTitel}

%% -- Für Kopfzeile und TOC Eintrag
%% --
\markleft{\textsc{\AutorenBeitrag}}		% auf den geraden Seiten erscheinen die Autoren
\markright{\textsc{\ShortTitel}}		% auf den ungeraden Seiten der kurze Titel
%% 							
\addtocontents{toc}{\textsc{\AutorenBeitrag}}
\addtocontents{toc}{}

%% -- Titelseite des Beitrags
%% --
\begin{center}
\textsc{\Large \AutorenBeitrag}
\end{center}
	\vspace{1cm}
\begin{center}
	\includegraphics[width=8cm, height=8cm, keepaspectratio=true]{./content/Rom-Artikel-01-Cover.jpg}
\end{center}

%% -- Würde keine Extraseite daraus machen
%% --
%\newpage

%% --
%% -- ab hier die Sammlung der LIT-Zitate
\begin{refsection}

%% -- Begin des eigenen Artikels
%% --
\section*{Eine erste Überschrift}
% 
Da es sich um kurze Beiträge handelt, kann man aus meiner Sicht auf Unterverzeichnisse via
\verb|\subsection{Überschrift}| verzichten.
Wer es dennoch machen will bitte beachten, dass ein solcher Abschnitt dann mindestens drei Absätze enthalten sollte, die wiederum aus mindestens fünf Sätzen bestehen.

Bei der Eingabe bitte folgendes beachten.

\begin{itemize}[nosep, topsep=1em]
\item
Text, der in Hochkommata (Gänzefüßchen) eingeschlossen ist bitte mittels \verb|\enquote{Text}|
eingegeben, \dh \enquote{Text}; dann erscheinen auch die richtigen.

\item
Auszeichnungen werden nicht fett mittels \verb|\textbf{Text}|, sondern mit \verb|\emph{Text}|
eingegeben, \dh \emph{Text}.
Bitte nie \verb|\textit{Text}| verwenden.

\item
Es ist ein Unterschied, ob man d.h. schreibt oder korrekt \dh
Für die Abkürzungen mal in die Datei \texttt{Rom-Abkuerzungen.tex} reinsehen und auch den Duden konsultieren.

\item
Für Aufzählungen, \dh Listen, hat sich das Paket \texttt{enumitem} mit den vielen Möglichkeiten bewährt.
Will man etwa römische kleine Zahlen haben: \verb|\begin{enumerate}[(i)]| macht es.

\begin{enumerate}[(i), nosep]
\item
Etwa hier und dann
\item
Hier nochmals
\end{enumerate}

Einfach mal diese Datei im Original ansehen.
Vor allem, wie man die Listen kompakter bekommt.

\item
Literatur kann man einfach mit der \LaTeX{} Umgebung \texttt{thebibliography} eingeben oder mit Hilfe von  \texttt{bibtex} und \texttt{biblatex}.
Auf jeden Fall sollten die Zitate korrekt sein, \dh bei Verweise auf die mathematsiche Artiekl etwa die Abkürzungen der Zeitschriften.
Dazu bitte das \emph{Zentralblatt für Mathematik} nutzen; siehe \href{https://zbmath.org}{https://zbmath.org}.
Weiters findet sich in meinem \LaTeX-Tipp zur Literaturverwaltung. 

\end{itemize}

Der nun folgende Text stammt nicht von mir, da ich kein Schachspieler bin sondern Go bevorzuge.
Zu Go siehe etwa \href{https://bit.ly/3CEdJom}{diesen Artikel auf Wikipedia}.

Literatur zu \TeX{}: \textcite{kohm:2020}, \textcite{voss:2012a,voss:2017b}, \textcite{latextipps5}.

\subsection{Etwas zu Schach}

Schach ist ein Brettspiel und wird auf einem quadratischen Brett gespielt. Dies ist in 64 Felder eingeteilt.

Am Anfang bekommt jeder Spieler 16 Figuren. Dies sind:
\begin{enumerate}[label=\textbullet, nosep]  % enger zusammen
  \item 
  acht Bauern,
  \item 
  zwei Türme,
  \item 
  zwei Springer,
  \item 
  zwei Läufer,
  \item 
  eine Dame,
  \item 
  und ein König.
\end{enumerate}

\subsection{Eröffnung}
Den Beginn einer Schachpartie nennt man Eröffnung. 
Diese sind zumeist sehr ausführlich analysiert. 
Viele Schachprogramme haben eine Eröffnungsbibliothek mit mehreren tausend Zügen. 
Eine Eröffnung könnte beispielsweise so aussehen
%
\begin{enumerate}[--, nosep]
  \item e4 e5
  \item Lc4 d6
  \item Df3 Sc6
  \item Dxf6\#
\end{enumerate}

und man nennt diese Eröffnung das \emph{Schäfermatt} (Zugfolge hier ist aber falsch)

%%% --------------------------------
%%%  -- Was
%%% --------------------------------
%\par\medskip
%\begin{minipage}[c][1.0\height][r]{.95\linewidth}
%\begin{center}
%\newchessgame
%\mainline{1. e4 e5 2. Bc4 d6 3. Qh5 Nf6 4. Qf7}
%\xskakset{moveid=3w}%
%\chessboard[setfen=\xskakget{nextfen}]\\[1ex]
%Position nach Zug 3 \, \xskakget{lan}
%\end{center}
%Dann zieht Weiß auf f7 und Schwarz ist matt.
%Wer es genauer wissen will, bitte \href{https://ctan.org/pkg/xskak}{https://ctan.org/pkg/xskak} nachsehen \bzw Manual lesen. 
%\end{minipage}
%\par\medskip
%% ----------------------------
%


\subsection{Verlauf des Spieles}
%
Ein Schachspiel unterteilt sich in drei Abschnitte:
%
\begin{description}
  \item[Eröffnung] 
  Dies ist der Anfang einer Partie. 
  Man versucht, zunächst alle Figuren optimal zu platzieren.
  
  \item[Mittelspiel] 
  Hier kommt es sehr auf Strategie und Taktik an. 
  Jeder Spieler versucht, das Spiel zu seinen Gunsten zu steuern.
  
  \item[Endspiel] 
  Wenn dann nur wenige Figuren auf dem Brett verblieben sind, spricht man vom Endspiel. 
  Meist wird versucht, einen Bauern in eine höherwertige Figur umzuwandeln und den König Matt zu setzen.
\end{description}
%%
Die Rangliste sieht wie folgt aus (etwas veraltet):
\begin{center}
  \begin{tabular}{llc}
    Rang & Name              & Rating \\\hline
    1    & Garry Kasparov    & 2817   \\
    2    & Viswanathan Anand & 2774   \\
    3    & Wladimir Kramnik  & 2764
  \end{tabular}
\end{center}
%% \vref bitte verwenden
Literatur hierzu kann nachgelesen in \textcite{brunel:1999}. 
Zum Schluß noch ein kleines Bild, siehe \vref{fig:label1} und etwas \textbf{fette} oder \emph{kursive} Schrift.

\subsection{Unterabschnitt}
%%
\blindtext[2]

%%
\section*{Ein neuer Hauptabschnitt}
\subsection{Mit einem neuen Unterabschnitt}
\lipsum[1-3]

\subsection{Noch ein Unterabschnitt}
\lipsum[4-6]

\begin{figure}[t]
  \begin{center}
  \includegraphics[width=55mm]{./content/Rom-Artikel-01-Cover.jpg}
  \caption{Bildunterschift 1}\label{fig:label1}		%\vspace{-3mm}
  \end{center}
\end{figure}

%% --
%% -- LitVerzeichnis des Beitrags
%\nocite{*}
% \addbibresource{./bib/name.bib}
\printbibliography
\end{refsection}

%\bibliography{article} -> falsch siehe Masterfile
	%% Das schlechte Muster

%% -- Ansonsten per Teilnehmer
%\input{./content/Rom-Beitrag-Name/Name-Artikel}

%% -- Backmattter
\backmatter

%% -- Literturverzeichnis
%% -- Jeder Beitrag hat seine eigenes Literaturverzeichnis
%% --
%\nocite{voss:2012a,kohm:2020,voss:2017b,latextipps5}
%\printbibliography

%% --
\end{document} 